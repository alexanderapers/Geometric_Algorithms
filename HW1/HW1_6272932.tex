\documentclass{article}
\usepackage{xspace}
\usepackage{fancyhdr}
\usepackage{graphicx}
\usepackage{amsmath,amssymb,amsthm}
\usepackage{microtype}
\usepackage{enumitem}
\usepackage{ifthen}
\usepackage{totcount}
\usepackage{algpseudocode}
\usepackage[usenames,usedvpinames]{xcolor}
\usepackage[most]{tcolorbox}
\usepackage[a4paper, margin=1in]{geometry}
\usepackage{parskip}

\graphicspath{ {./figures/} }

%--------------------------------------------------------------------------------

\newcommand{\edition}{2022-2023}
\newcommand{\deadline}{25 November 2022, 13:15}
\newcommand{\hwNum}{1}

%--------------------------------------------------------------------------------
% Style stuff
\pagestyle{fancy}
\lhead{Geometric Algorithms (INFOGA \edition)}
\chead{id: 6272932}
\rhead{\mytitle}
\lfoot{INFOGA \edition}
\cfoot{}
\rfoot{\thepage}
\renewcommand{\headrulewidth}{0.4pt}
\renewcommand{\footrulewidth}{0.4pt}
\renewcommand{\familydefault}{\sfdefault}

%--------------------------------------------------------------------------------
% Question macro's

\newcommand{\withpoints}[1]{%
  \addtocounter{pointscounter}{#1} \printpoints{#1}
}
\newcommand{\printpoints}[1]{%
   \ifthenelse{#1 = 0}
              {}
              {\textit{(#1 points)}}\mbox{}
}



\newenvironment{question}[1][0]{\begin{tcolorbox}[parbox=false
                                      ,breakable=true
                                      ,enhanced jigsaw
                                      ,title=\bfseries Question
                                      \stepcounter{questionscounter}\arabic{questionscounter}
                                      \withpoints{#1}]
                         }
                         {\end{tcolorbox}}
\newenvironment{subquestions}{\begin{enumerate}[leftmargin=*
                                        ]}
                             {\end{enumerate}}
\newcommand{\subquestion}[1][0]{\item \withpoints{#1}}

\newtotcounter{questionscounter}
\newtotcounter{pointscounter}

\newcommand{\numquestions}{\total{questionscounter}}
\newcommand{\numpoints}{\total{pointscounter}}

%--------------------------------------------------------------------------------
\newcommand{\myremark}[3]{\textcolor{blue}{\textsc{#1 #2:}} \textcolor{red}{\textsf{#3}}}
% \renewcommand{\myremark}[3]{}
\newcommand{\frank}[2][says]{\myremark{Frank}{#1}{#2}}

%--------------------------------------------------------------------------------
% Theorem Environments
\newtheorem{theorem} {Theorem}
\newtheorem{lemma}[theorem] {Lemma}
\newtheorem{corollary}[theorem] {Corollary}
\newtheorem{problem}[theorem] {Problem}
\newtheorem{observation}[theorem] {Observation}
\newtheorem{claim}[theorem] {Claim}
\newtheorem{invariant}[theorem] {Invariant}

%--------------------------------------------------------------------------------
% Otherwise useful Macro's
\newcommand{\eps}{\ensuremath{\varepsilon}\xspace}
\DeclareMathOperator{\argmin}{argmin}
\DeclareMathOperator{\argmax}{argmax}

\newcommand{\mkmbb}[1]{\ensuremath{\mathbb{#1}}\xspace}
\newcommand{\R}{\mkmbb{R}}

\newcommand{\mathfunc}[1]{\ensuremath{\mathit{#1}}\xspace}

%--------------------------------------------------------------------------------

\algrenewcommand{\algorithmiccomment}[1]{\hfill// #1}
\newcommand{\CH}{\mathcal{C}\mathcal{H}}

% The Actual document

\newcommand{\mytitle}{Homework Exam \hwNum\xspace\edition}
\title{\mytitle}
\date{\textbf{Deadline: } \deadline}
\author{Alexander Apers \\ 6272932}


\begin{document}
\maketitle
\thispagestyle{fancy}
  \noindent
  This homework exam has \numquestions\xspace question for a total of
  \numpoints\xspace points. You can earn an additional point by a careful
  preparation of your hand-in: using a good layout, good spelling, good
  figures, no sloppy notation, no statements like ``The algorithm runs in
  $n\log n$.''  (forgetting the $O(..)$ and forgetting to say that it concerns
  time), etc. Use lemmas, theorems, and figures where appropriate.

  \begin{question}[9]
    Let $r \in \R^2$ be a ``red'' point, and let $B$ be a set of $n$
    ``blue'' points in $\R^2$. You can assume that the points are in
    general position; meaning that no two points have the same
    $x$-coordinate or the same $y$-coordinate, and that no three
    points lie on a line. A triangle is ``bichromatic'' when its
    vertices are either red or blue, and it has at least one vertex of
    either color. Develop an $O(n\log n)$ time algorithm to find a
    maximum area ``bichromatic'' triangle $\Delta^*$ on $\{r\},B$.

    \textbf{Hint: } You can use the following fact. A function
    $f [1..n] \to \R$ is \emph{unimodal} if (and only if) it has a
    single (local) maximum. A maximum of $f$ can be computed in
    $O(T\log n)$ time, where $T$ is the time it takes to evaluate a
    single value $f(i)$ with $i \in [1..n]$. In particular, using the
    following function \Call{TernarySearch}{$[1,n], f$}:

    \begin{algorithmic}
    \Function{TernarySearch}{$[a..b], f$}
      \State $n \gets b - a$
      \If{ $n < 3$ }
        evaluate $f(i)$ for each $i \in [a..b]$ and return $\max_i f(i)$
      \Else
        \State $m_1 \gets a + \lfloor n/3  \rfloor$ ; $m_2 \gets a + \lfloor 2n/3 \rfloor$
        \If{ $f(m_1) < f(m_2)$ }
          \Call{TernarySearch}{$[m_1..,b], f$}
        \Else
          \Call{TernarySearch}{$[a..m_2], f$}
        \EndIf
      \EndIf
    \EndFunction
    \end{algorithmic}
  \end{question}

  \pagebreak

  \section*{Solution to Question 1}
  \subsection*{Definitions}
  Let $A$ denote the set $\{r\} \cup B$ \\
  Let $\CH(A)$ denote the convex hull of point set $A$ (including interior and edges) \\
  Let CH($A$) denote the set of vertices that make up the edges of $\CH(A)$

  \subsection*{Geometric Observations}
  Since there is only one red point, it is clear that all bichromatic triangles contain exactly one red point and two blue points.
  We start with the following observation. \\

  \begin{lemma}
  The maximum area bichromatic triangle must necessarily consist of 2 blue vertices, $b_1, b_2$ such that $b_1, b_2 \in$ CH($A$).
  \end{lemma}

  \begin{proof}
      Suppose for contradiction that $b_1 \notin$ CH($A$). We denote the maximum area triangle $\triangle rb_1b_2$.
      % We denote this triangle $\triangle rb_{1}b_{2}$, with the point $b_{2}$ not on the boundary of $\CH(A)$.
      % We will show that the area of $\triangle rb_1b_2$ can be increased by using a vertex $x \in$ CH($A$) instead of $b_1$.
      Imagine the line $l$ parallel to the line going through $r$ and $b_{2}$ which goes through the point $b_{1}$.
      If we replace $b_{1}$ with any point on the other side of $l$ compared to where $r$ and $b_{2}$ are, the area of $\triangle rb_1b_2$ will increase as this point lies farther from the segment $\overline{rb_2}$ than $b_1$.
      $l$ must cross the boundary of $\CH(A)$ in 2 locations since it goes through point $b_1 \in \CH(A)$ but $b_1 \notin$ CH($A$).
      Therefore, there must be a least some point $x \in$ CH($A$) further away from $\overline{rb_2}$ than $b_1$ to connect the two points where $l$ crosses $\CH(A)$.
      Therefore, it's guaranteed that $b_{1}$ can be replaced by at least one point $x \in$ CH($A$) so that the area of $\triangle rxb_2 > \triangle rb_1b_2$.
      Hence, $\triangle rb_{1}b_{2}$ is not a maxmium area triangle. This is a contradiction, since we assumed $\triangle rb_1b_2$ had maximum area.
      We conclude that $b_1 \in$ CH($A$) to be a maximum area triangle. 
  \end{proof}

  Similarly, it can be proven that $b_2 \in$ CH($A$).

  \vspace{0.5cm}
  \begin{lemma}
      Given point $r$ and a point $b_1 \in$ CH($A$), and line $l$ going through them, 
      TriangleArea($\triangle rb_1b_2$) is a unimodal function for $b_2 \in$ CH($A$) on one side of $l$.
  \end{lemma}

  \begin{proof}
      The area of $\triangle rb_1b_2$ is defined as $d(r, b_1) \cdot d(b_2, \overline{rb_1})$. The only input to the function which changes is the point $b_2$. Therefore, the area of the triangle is determined by $b_2$.
      If the set of vertices in CH($A$) on one side of $l$ is not empty, the distances $d(b_2, \overline{rb_1})$ have to be (1) monotonically non-decreasing, then (2) maximal, and finally (3) monotonically non-increasing, in that order when iterating through them in clockwise order.
      If this is not the case you can find 3 consecutive points $x, y, z \in$ CH($A$) such that $d(y, \overline{rb_1}) < d(x, \overline{rb_1})$ and $d(y, \overline{rb_1}) < d(z, \overline{rb_1})$.
      If this were the case then CH($A$) would contain vertices that create a non-convex polygon. 
      Therefore, TriangleArea($\triangle rb_1b_2)$ is a unimodal function for $b_2 \in$ CH($A$) on one side of $l$.
  \end{proof}

  \subsection*{Algorithm Description}
  % TODO add reference to algorithm
  Finding the vertices that make up the boundary of the convex hull can be done in $O(n\log n)$ time complexity using the algorithm discussed in class.
  These vertices will be given in clockwise (or counterclockwise) order. 
  
  The algorithm iterates through all points $b_1 \in$ CH($A$). For every point, the line $l$ is determined that goes through $r$ and $b_1$.
  Then, CH($A$) is partitioned into a subset with points left of $l$ and a subset with points right of $l$. 
  The consecutive points $i, j$ such that $i$ and $j$ are on opposite sides can be found using a simple BinarySearch. 
  At every step, half of the input can be eliminated by evaluating which side of the line $l$ the middle point in the input is on. 
  Since the AreaTriangle function is unimodal on both sides of the line, we apply TernarySearch once for both sides of $l$.
  The largest of these two triangles is the maximum area triangle that uses the vertices $b_1$ and $r$. 
  Since this is done for all $b_1 \in$ CH($A$), we simply have to keep track of the largest area triangle, which is returned at the end. \\

  \pagebreak

  \begin{algorithmic}
    \Function{MaxAreaBichromaticTriangle}{$r, B,$ CH($A$)} 
      \State MaxArea $\gets 0$ 
      \State MaxAreaTriangle $\gets \triangle$ 
      \For{$k \gets 1$ to $m$}
        \State $b_2 \gets$ CH($A)[k]$ 
        \State $l \gets$ line through $r$ and $b_1$ 
        % \State \algorithmiccomment{find 2 consecutive vertices $i, j \in$ CH($A$) such that $i$ and $j$ are on opposite sides of $l$.} 
        \State $i, j \gets$ BinarySearch(CH($A$), $l$) 
        \State left $\gets$ subset of CH($A$) left of $l$
        \State right $\gets$ subset of CH($A$) right of $l$
        \State $b_{2\text{left}} \gets$ TernarySearch(left, AreaTriangle) 
        \State $b_{2\text{right}} \gets$ TernarySearch(right, AreaTriangle) 
        \State $b_{\text{max}} \gets$ max($b_{2\text{left}}, b_{2\text{right}}$) 
        \If{AreaTriangle($\triangle rb_1b_{\text{max}}) \geq$ MaxArea}
          \State MaxArea $\gets$ AreaTriangle($\triangle rb_1b_{\text{max}})$ 
          \State MaxAreaTriangle $\gets \triangle rb_1b_{\text{max}}$ 
        \EndIf
      \EndFor
    \EndFunction
  \end{algorithmic}

  \subsection*{Algorithm Correctness}
  It is guaranteerd that the largest area bichromatic triangle is discovered by the algorithm.
  This is because the two TernarySearch calls find the vertex $b_2$ such that $\triangle rb_1b_2$ has maximum area for every $b_1 \in$ CH($A$). 
  TernarySearch can be applied using the AreaTriangle function and the given input due to \textbf{Lemma 2}.
  \textbf{Lemma 2.} shows that the two blue points need to be on the edge of the convex hull to be the maximum area triangle.


    \subsection*{Running Time Analysis}
    The outer loop loops over every vertex $b_1 \in$ CH($A$).
    There is an $O(n)$ number of vertices in CH($A$). 
    Binary search to find the two consecutive vertices in CH($A$) that are on opposite sides of $l$ can be done in $O(\log n)$.
    Applying TernarySearch twice on points in CH($A$) on either side of $l$ can be done in $O(\log n)$ time complexity, since evaluating AreaTriangle (calculating the area of a triangle with given vertices) can be done in  $O(1)$ time complexity.
    This means that the whole algorithm will run in $O(n\log n)$ time complexity.


\end{document}
