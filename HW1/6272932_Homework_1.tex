\documentclass[a4paper]{article}

\usepackage[utf8]{inputenc}
\usepackage[margin=2.54cm]{geometry}

\usepackage{amsmath,amssymb,amsthm}
\usepackage{commath,mathtools}
\usepackage{parskip}
\usepackage{graphicx}
\usepackage{xcolor}
\usepackage{subcaption}
\usepackage[english]{babel}
\usepackage{fancyhdr}
\usepackage{lmodern}
\usepackage[T1]{fontenc}
\usepackage{csquotes}
\usepackage{lipsum}
\usepackage{comment}
\usepackage{authblk}
\usepackage[bookmarksnumbered]{hyperref}
% \usepackage{algorithm}
\usepackage[linesnumbered,ruled]{algorithm2e}

% 20 most commonly used fonts:
% https://www.draketo.de/anderes/latex-fonts.html#orge83df8b

% roman: \rmdefault, sans serif: \sfdefault
\renewcommand{\familydefault}{\sfdefault}

\pagestyle{fancy}
\newtheorem{theorem}{Theorem}
\newcommand{\CH}{\mathcal{C}\mathcal{H}}

\title{\huge{Geometric Algorithms \\ Homework Exam 1}}
\author{Alexander Apers \\ 6272932}
\affil{\small{Utrecht University\\ \href{a.p.apers@uu.nl}{a.p.apers@uu.nl}}}
\date{\today}

\setcounter{secnumdepth}{2}

\begin{document}
    \maketitle
    % \tableofcontents

    % TODO add question
    % TODO add figures
    \section*{Question 1}
    \subsection*{Definitions}
    Let $A$ denote the set $\{r\} \cup B$ \\
    Let $\CH(A)$ denote the convex hull of point set $A$ (including interior and edges) \\
    Let CH($A$) denote the set of vertices that make up the edges of $\CH(A)$

    \subsection*{Geometric Observations}
    Since there is only one red point, it is clear that all bichromatic triangles contain exactly one red point and two blue points.
    We start with the following observation. \\

    \begin{theorem}
    The maximum area bichromatic triangle must necessarily consist of 2 blue vertices, $b_1, b_2$ such that $b_1, b_2 \in$ CH($A$).
    \end{theorem}

    \begin{proof}
        Suppose for contradiction that $b_1 \notin$ CH($A$). We denote the maximum area triangle $\triangle rb_1b_2$.
        % We denote this triangle $\triangle rb_{1}b_{2}$, with the point $b_{2}$ not on the boundary of $\CH(A)$.
        We will show that the area of $\triangle rb_1b_2$ can be increased by using a vertex $x \in$ CH($A$) instead of $b_1$.
        Imagine the line $l$ parallel to the line going through $r$ and $b_{2}$ which goes through the point $b_{1}$.
        If we replace $b_{1}$ with a point on the other side of $l$ compared to where $r$ and $b_{2}$ are, the area of $\triangle rb_1b_2$ will increase.
        $l$ must cross the boundary of the convex hull in 2 locations since $b_1 \in \CH(A)$ but $b_1 \notin$ CH($A$).
        Therefore, it's guaranteed that $b_{1}$ can be replaced by at least one point $x \in$ CH($A$) so that the area of $\triangle rxb_2 > \triangle rb_1b_2$.
        Hence, $\triangle rb_{1}b_{2}$ is not a maxmium area triangle. This is a contradiction. 
        We conclude that $b_1 \in$ CH($A$) to be a maximum area triangle. 
    \end{proof}

    Similarly, it can be proven that $b_2 \in$ CH($A$).

    \vspace{0.5cm}
    \begin{theorem}
        Given point $r$ and a point $b_1 \in$ CH($A$), and line $l$ going through them, 
        TriangleArea($\triangle rb_1b_2$) is a unimodal function for $b_2 \in$ CH($A$) on one side of $l$.
    \end{theorem}

    \begin{proof}
        The area of $\triangle rb_1b_2$
    \end{proof}

    \subsection*{Algorithm Description}
    % TODO add reference to algorithm
    Finding the vertices that make up the boundary of the convex hull can be done in $O(n\log n)$ time complexity. 
    These vertices will be given in clockwise (or counterclockwise) order. 

    \begin{algorithm}[H]
        \SetKwInOut{Input}{Input}
        \SetKwInOut{Output}{Output}
    
        \Input{$r, B,$ CH($A$) as an array sorted clockwise of size $m$}
        \Output{Maximum Area Bichromatic Triangle}
        MaxArea $\gets 0$ \\
        MaxAreaTriangle $\gets \triangle$ \\
        \For{$k \gets 1$ \KwTo $m$}
        {
            $b_1 \gets$ CH($A)[k]$ \\
            $l \gets$ line through $r$ and $b_1$ \\
            \tcp{find 2 consecutive vertices $i, j \in$ CH($A$) such that $i$ and $j$ are} 
            \tcp{on opposite sides of $l$.} 
            $i, j \gets$ BinarySearch($l$, CH($A$)) \\
            $b_{2\text{left}} \gets$ TernarySearch(CH($A$)$[1,...,i]$, AreaTriangle) \\
            $b_{2\text{right}} \gets$ TernarySearch(CH($A$)$[j,...,m]$, AreaTriangle) \\
            $b_{\text{max}} \gets$ max($b_{2\text{left}}, b_{2\text{right}}$) \\
            \If {\upshape AreaTriangle($\triangle rb_1b_{\text{max}}) \geq$ MaxArea}
            {
                MaxArea $\gets$ AreaTriangle($\triangle rb_1b_{\text{max}})$ \\ 
                MaxAreaTriangle $\gets \triangle rb_1b_{\text{max}}$ 
            }
        }
        \Return MaxAreaTriangle
        \caption{Maximum Area Bichromatic Triangle}
    \end{algorithm}

    \subsection*{Algorithm Correctness}


    \subsection*{Running Time Analysis}
    The outer loop loops over every vertex $b \in$ CH($A$).
    This is an $O(n)$ number of vertices in CH($A$). 
    Binary search to find the two consecutive vertices that are on opposite sides of $l$ can be done in $O(\log n)$.
    Applying TernarySearch twice on points on the boundary of the convex hull can be done in $O(\log n)$ time complexity, since evaluating AreaTriangle (calculating the area of a triangle with given vertices) can be done in  $O(1)$ time complexity.
    This means that the whole algorithm will run in $O(n\log n)$ time complexity.





\end{document}